\RequirePackage{plautopatch}
\RequirePackage[l2tabu, orthodox]{nag}

\documentclass[platex,dvipdfmx]{jlreq}			% for platex
% \documentclass[uplatex,dvipdfmx]{jlreq}		% for uplatex
\usepackage{graphicx}
\usepackage{bxtexlogo}
\usepackage{braket}
% \usepackage{amsmath}

\usepackage{hyperref}
\usepackage{color}
\usepackage{siunitx}
\usepackage{indentfirst}
\usepackage{here}
\usepackage{physics}
\usepackage{amsmath,amssymb}
% 図形
\usepackage{tikz}
\usepackage{url}

\title{グローバーのアルゴリズム}

\author{9BSP1118村岡海人}
\date{\today}

\begin{document}
    \maketitle

    \section{概要}
    このグローバーのアルゴリズムは以下のような流れで行う。
    $N$個のデータに対して、$\Omega(\sqrt{N})$回の計算量で回を見出すことができる。
    古典的な探索アルゴリズムにも同じ計算量を持つ2分探索があるが、2分探索アルゴリズムは事前にソートされているデータを扱うため、ソートされていないデータの探索アルゴリズムではグローバーのアルゴリズムの方が高速である。

    このグローバーのアルゴリズムは以下のような流れで行う。
    $n$を量子ビット数とすると、$N = 2^n$の要素からなるデータベースから$M$個の解を探索する問題を考え、要素のラベルを$n$桁のビット列$x = x_1 \cdots x_n$とする。

    \begin{itemize}
        \item 全ての状態の重ね合わせ状態$\ket{s} = \frac{1}{\sqrt{N}} \sum_x \ket{x}$を用意する 
        \item 演算子$U_w$(解に対する反転操作)を作用させる 
        \item 演算子$U_s$($\ket{s}$を対称軸にした反転操作)を作用させる 
        \item 2、3を$k$回繰り返す 
    \end{itemize}

    \section{アルゴリズムの流れ}
    まず初めに、全ての状態の重ね合せ状態$\ket{s} = \frac{1}{\sqrt{N}} \sum_x \ket{x}$を用意する。
    % 検索したい値を$w$として、
    初期状態$\ket{0}^{\otimes n}$に対して全ての量子ビットにアダマール演算をかけると、

    \begin{equation}
        \begin{split}
        \ket{s} &= H^{\otimes n} \ket{0}^{\otimes n} \\ 
        &= (H \otimes \cdots \otimes H)\ket{0\cdots0} \\
        &= \frac{1}{(\sqrt{2^n})} (\ket{0} + \ket{1})\otimes\cdots\otimes (\ket{0} + \ket{1}) \\
        &= \frac{1}{\sqrt{2^n}} \left( \ket{00 \cdots 00} + \ket{00 \cdots 01} + \cdots + \ket{11 \cdots 10} + \ket{11 \cdots 11}\right) \\
        \ket{s} &= \frac{1}{\sqrt{2^n}} \sum^{2^n - 1}_{x = 0} \ket{x}
        \end{split}
    \end{equation}

    のように計算できる。

    次に解に対する反転操作を作用させる。入力$\ket{x}$に対して$x$が解なら、$-1$を掛けて位相を反転し、解でないなら$1$を掛ける。つまり、ゲートを以下のように定義する。
    検索したい値を$w$として、

    \begin{equation}
        \left\{
        \begin{array}{l}
        U_w \ket{x} = \ket{x} (x \neq w) \\
    U_w \ket{w} = - \ket{w}
        \end{array}
    \right.
    \end{equation}
    
    \begin{equation}
        U_w = I -2 \sum_{w \in \text{解}} \ket{w}\bra{w}
    \end{equation}
    
    
    となる。これを用いると、$\ket{s}$は、
    \begin{equation}
    \begin{split}
        U_w\ket{s} &= \frac{1}{\sqrt{2^n}} \sum^{2^n - 1} _{x = 0, x \neq w} U_w \ket{x} + \frac{1}{\sqrt{2^n}}U_w \ket{x}\\
        &= \frac{1}{\sqrt{2^n}}\sum^{2^n - 1}_{x = 0, x \neq w} \ket{x} - \frac{1}{\sqrt{2^n}}\ket{w}\\
        &= \left( \ket{s} - \frac{1}{\sqrt{2^n}}\ket{w} \right) - \frac{1}{\sqrt{2^n}}\ket{w} \\
        U_w\ket{s} &= \ket{s} - \frac{2}{\sqrt{2^n}} \ket{w} \label{eq:f-inf}
    \end{split}
    \end{equation}
    のように計算できる。

    最後に$\ket{s}$を対称軸にした反転操作$U_s$を定義する。
    \begin{equation}
        \begin{split}
            U_s &= 2 \ket{s}\bra{s} - I = H^{\otimes n}(2 \ket{0 \cdots 0}\bra{0 \cdots 0} - I)H^{\otimes n}
        \end{split}
    \end{equation}


    \begin{equation}
        \left\{
            \begin{array}{l}
                U_s \ket{x} = 2 \bra{s} x \ket{s} - \ket{x} = \frac{2}{\sqrt{2^n}} \ket{x} \\
                U_s \ket{s} = 2 \bra{s} s \ket{s} - \ket{s} = \ket{s}
            \end{array}
        \right.
    \end{equation}


    式(4)より、$U_s$を作用させると、

    \begin{equation}
    \begin{split}
            U_s \ket{s} - \frac{2}{\sqrt{2^n}} U_s \ket{w} &= \ket{s} - \frac{2}{\sqrt{2^n}} \left( \frac{2}{\sqrt{2^n}} \ket{s} - \ket{w} \right) \\
        &= \frac{2^n - 4}{2^n} \ket{s} + \frac{2}{\sqrt{2^n}} \ket{w} \\
        &= \frac{2^n - 4}{2^n \sqrt{2^n}} \sum^{2^n - 1}_{x = 0, w \neq w} \ket{x} + \left( \frac{2^n - 4}{2^n \sqrt{2^n}} + \frac{2}{\sqrt{2^n}} \right) \ket{w} \\
        &= \frac{2^n - 4}{2^n \sqrt{2^n}} \sum^{2^n - 1}_{x = 0, x \neq w} \ket{x} + \frac{3 \cdot 2^n - 4}{2^n \sqrt{2^n}} \ket{w}
    \end{split}
    \end{equation}
    ように計算できる。

    $\ket{s}$の時の状態では、$\ket{w}$を測定すると、確率は$\frac{1}{2^n}$となる。式(2.7)から確率が上昇していることがわかる。この確率を増幅させる操作のことを、振幅増幅と呼ぶ。グローバーのアルゴリズムは、この振幅増幅を複数回行うことにより、$\ket{w}$の確率を1に近づける。


    \section{図を使用しての説明}
$\ket{w}$に直行するベクトル$\ket{w^{\perp}}$を用いた平面を考えると、以下のような状態が得られる。
$$
    \ket{w} = \frac{1}{\sqrt{N - M}} \sum^{2^n - 1}_{x = 0, w \neq 0} \ket{x}
$$
$$
    \ket{w^{\perp}} = \frac{1}{\sqrt{M}} \ket{w}
$$
全ての状態の重ね合せ状態$\ket{s}$は次のように表すことができるので、2次元平面ベクトルであることがわかる。
\begin{equation}
    \ket{s} = \sqrt{\frac{N - M}{N}}\ket{w^{\perp}} + \sqrt{\frac{M}{N}}\ket{w}
\end{equation}

全ての状態の重ね合わせ状態$\ket{s}$は次のように表せるので、この2次元平面内ベクトルであることがわかる。
式(3.1)より、$\cos{\frac{\theta}{2}} = \sqrt{\frac{N - M}{N}}, \sin{\frac{\theta}{2}} = \sqrt{\frac{M}{N}}$を満たす角$\theta$を用いれば、
\begin{equation}
    \ket{s} = \cos{\frac{\theta}{2}}\ket{w^{\perp}} + \sin{\frac{\theta}{2}}\ket{w}
\end{equation}
と表すことができる。これを図示すると、図1のようになる。

\begin{figure}[H]
    \centering
    \begin{tikzpicture}
        \draw[->,>=stealth,semithick] (-0.5,0)--(4,0)node[above]{$\ket{w^{\perp}}$}; %x軸
        \draw[->,>=stealth,semithick] (0,-0.5)--(0,3.5)node[right]{$\ket{w}$}; %y軸
        \draw (0,0)node[below left]{O}; %原点
        \draw[->] (0,0)--(3, 1.5)node[right]{$\ket{s}$};
        \draw [-](1,0) arc (0:20:1.4);
        \coordinate (thrta_0) at (1,0) node at (thrta_0) [above right] {$\theta / 2$};
    \end{tikzpicture}
    \caption{全ての状態の重ね合わせ状態$\ket{s}$}
    \label{fig:rootate}
\end{figure}

次に、$\ket{s}$に$U_w$をかけることにより、$\ket{w^{\perp}}$を軸に反転すると、図2のようになる。
\begin{figure}[H]
    \centering
    \begin{tikzpicture}
        \draw[->,>=stealth,semithick] (-0.5,0)--(4,0)node[above]{$\ket{w^{\perp}}$}; %x軸
        \draw[->,>=stealth,semithick] (0,-0.5)--(0,3.5)node[right]{$\ket{w}$}; %y軸
        \draw (0,0)node[below left]{O}; %原点
        \draw[->, dotted] (0,0)--(3, 1.5)node[right]{$\ket{s}$};
        \coordinate (thrta_0) at (1,0) node at (thrta_0) [above right] {$\theta / 2$};
        \draw [-](1,0) arc (0:20:1.4);
        
        \draw[->] (0,0)--(3, -1.5)node[right]{$U_w \ket{s}$};
        \coordinate (thrta_1) at (1,0) node at (thrta_0) [below right] {$\theta / 2$};
        \draw [-](1,0) arc (0:-20:1.4);
    \end{tikzpicture}
    \caption{$\ket{s}$に$U_w$を作用}
    \label{fig:rootate}
\end{figure}

最後に、$U_s$を作用させることにより、$\ket{s}$を軸に$U_w \ket{s}$を反転させると、図3のようになる。

\begin{figure}[H]
    \centering
    \begin{tikzpicture}
        \draw[->,>=stealth,semithick] (-0.5,0)--(4,0)node[above]{$\ket{w^{\perp}}$}; %x軸
        \draw[->,>=stealth,semithick] (0,-0.5)--(0,3.5)node[right]{$\ket{w}$}; %y軸
        \draw (0,0)node[below left]{O}; %原点
        % \draw[->, dotted] (0,0)--(3, 1.5)node[right]{$\ket{s}$};
        % \coordinate (thrta_0) at (1,0) node at (thrta_0) [above right] {$\theta / 2$};
        % \draw [-](1,0) arc (0:20:1.4);
        
        \draw[->, dotted] (0,0)--(3, -1.5)node[right]{$U_w \ket{s}$};
        \coordinate (thrta_1) at (1,0) node at (thrta_0) [below right] {$\theta / 2$};
        \draw [-](1,0) arc (0:-20:1.4);
        
        \draw[->] (0,0)--(2.25, 2.6)node[right]{$U_s U_w \ket{s}$};
        \coordinate (thrta_0) at (1,0) node at (thrta_1) [above right] {$\frac{3}{2}\theta$};
        \draw [-](1,0) arc (0:37.5:1.4);
    \end{tikzpicture}
    \caption{$U_w\ket{s}$に$U_s$を作用}
    \label{fig:rootate}
\end{figure}

以上より、平面ベクトル内で、角度$\theta$だけの回転が行われ、$\ket{w}$を測定する確率が上昇することがわかる。


\section{最適な$k$の見積もり}
最後に、$U_sU_w$を作用させる回数$k$について、最適な回数が幾つなのか調べる。
式(3.2)より、グローバーのアルゴリズムを1回施すと、
\begin{equation}
    U_sU_w\ket{s} = \cos{\frac{3}{2}\theta}\ket{w^{\perp}} + \sin{\frac{3}{2}\theta}\ket{w}
\end{equation}
となる。これを$k$回施すと、
\begin{equation}
    (U_sU_w)^k\ket{s} = \cos{\frac{2k+1}{2}\theta} \ket{w^{\perp}} + \sin{\frac{2k+1}{2}\theta} \ket{w}
\end{equation}
となる。これを用いて、最終的に$\ket{w}$の確率振幅を1にしたいので、

\begin{equation}
    \begin{aligned}
    \sin{\frac{2k + 1}{2} \theta} = 1\\
    \Leftrightarrow \frac{2k + 1}{2} \theta = \frac{\pi}{2}\\
    \therefore k = \frac{\pi}{2 \theta}
    \end{aligned}
\end{equation}

となる。よって、$\frac{2k + 1}{2}\theta$が$\frac{\pi}{2}$にもっとも近くなるときは、
\begin{equation}
    R = ClosestInteger(\frac{\pi}{2\theta} - \frac{1}{2})
\end{equation}
の時である。ここで、$ClosestInteger(...)$は$...$に最も近い整数を表す。

最後に、Rの上限を評価する。$\theta > 0$について成り立つ式、
\begin{equation}
    \frac{\theta}{2} \geq \sin{\frac{\theta}{2}} = \sqrt{\frac{M}{N}}
\end{equation}
を使うと、以下のように表すことができる。
\begin{equation}
    R \leq \left( \frac{\pi}{2\theta} - \frac{1}{2} \right) + 1 = \frac{\pi}{2\theta} + \frac{1}{2} \leq \frac{\pi}{4}\sqrt{\frac{N}{M}} + \frac{1}{2}
\end{equation}
つまり、Rは$O(\sqrt{N/M})$である。これにより、グローバーのアルゴリズムが$O(\sqrt{N})$で動作することがわかる。



 \begin{thebibliography}{2}
 \bibitem{QuantumDojo}\url{https://dojo.qulacs.org/ja/latest/notebooks/8.2_Grovers_algorithm.html}
 \bibitem{IBMbook}秀和システム:IBM Quantumで学ぶ量子コンピュータ
 \end{thebibliography}
\end{document}
