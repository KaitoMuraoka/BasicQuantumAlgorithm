\RequirePackage{plautopatch}
\RequirePackage[l2tabu, orthodox]{nag}

\documentclass[platex,dvipdfmx]{jlreq}			% for platex
% \documentclass[uplatex,dvipdfmx]{jlreq}		% for uplatex
\usepackage{graphicx}
\usepackage{bxtexlogo}
\usepackage{braket}
\usepackage{amsmath}

\title{グローバーのアルゴリズム}

\author{9BSP1118村岡海人}
\date{\today}

\begin{document}
    \maketitle

    \section{概要}
    このグローバーのアルゴリズムは以下のような流れで行う。
    $N$個のデータに対して、$\Omega(\sqrt{N})$回の計算量で回を見出すことができる。
    古典的な探索アルゴリズムにも同じ計算量を持つ2分探索があるが、2分探索アルゴリズムは事前にソートされているデータを扱うため、ソートされていないデータの探索アルゴリズムではグローバーのアルゴリズムの方が高速である。

    このグローバーのアルゴリズムは以下のような流れで行う。
    $n$を量子ビット数とすると、$N = 2^n$の要素からなるデータベースから$M$個の解を探索する問題を考え、要素のラベルを$n$桁のビット列$x = x_1 \cdots x_n$とする。

    \begin{itemize}
        \item 全ての状態の重ね合わせ状態$\ket{s} = \frac{1}{\sqrt{N}} \sum_x \ket{x}$を用意する 
        \item 演算子$U_w$(解に対する反転操作)を作用させる 
        \item 演算子$U_s$($\ket{s}$を対称軸にした反転操作)を作用させる 
        \item 2、3を$k$回繰り返す 
    \end{itemize}

    \section{アルゴリズムの流れ}
    まず初めに、全ての状態の重ね合せ状態$\ket{s} = \frac{1}{\sqrt{N}} \sum_x \ket{x}$を用意する。
    TODO:この内容を入れるかどうか検討する
    % 検索したい値を$w$として、
    初期状態$\ket{0}^{\otimes n}$に対して全ての量子ビットにアダマール演算をかけると、

    \begin{equation}
        \begin{split}
        \ket{s} &= H^{\otimes n} \ket{0}^{\otimes n} \\ 
        &= (H \otimes \cdots \otimes H)\ket{0\cdots0} \\
        &= \frac{1}{(\sqrt{2^n})} (\ket{0} + \ket{1})\otimes\cdots\otimes (\ket{0} + \ket{1}) \\
        &= \frac{1}{\sqrt{2^n}} \left( \ket{00 \cdots 00} + \ket{00 \cdots 01} + \cdots + \ket{11 \cdots 10} + \ket{11 \cdots 11}\right) \\
        \ket{s} &= \frac{1}{\sqrt{2^n}} \sum^{2^n - 1}_{x = 0} \ket{x}
        \end{split}
    \end{equation}

    のように計算できる。



\end{document}
