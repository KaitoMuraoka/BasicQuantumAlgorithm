\RequirePackage{plautopatch}
\RequirePackage[l2tabu, orthodox]{nag}

\documentclass[platex,dvipdfmx]{jlreq}			% for platex
% \documentclass[uplatex,dvipdfmx]{jlreq}		% for uplatex
\usepackage{graphicx}
\usepackage{bxtexlogo}
\usepackage{braket}
% \usepackage{amsmath}

\usepackage{hyperref}
\usepackage{color}
\usepackage{siunitx}
\usepackage{indentfirst}
\usepackage{here}
\usepackage{physics}
\usepackage{amsmath,amssymb}
% 図形
\usepackage{tikz}
\usepackage{url}

\title{レポートタイトル}

\author{9BSP1118 村岡海人}
\date{\today}
\begin{document}
\maketitle
\section{任意角度のスピンと1量子ビットの状態}

% ここからが本番
% 1量子ビットの任意のユニタリ回転ゲートは、x, y, z軸の周りにそれぞれ1量子ビットを回転させれば、任意のユニタリ回転ゲートができそうである。

任意のユニタリ回転ゲートを作成する。
x, y, z軸の周りに1量子ビットを回転させる行列は、次のように、パウリ演算子$X, Y, Z$から求められる。

\begin{equation}
    R_x(\theta) = e^{-i \theta X / 2} = \cos{\frac{\theta}{2}}I - i \sin{\frac{\theta}{2}} X = \begin{pmatrix}
        \cos{\frac{\theta}{2}} & -i \sin{\frac{\theta}{2}} \\
        -i \sin{\frac{\theta}{2}} & \cos{\frac{\theta}{2}} \\
    \end{pmatrix}
\end{equation}

\begin{equation}
    R_y(\theta) = e^{-i \theta Y / 2} = \cos{\frac{\theta}{2}} I - i \sin{\frac{\theta}{2}} Y =  \begin{pmatrix}
        \cos{\frac{\theta}{2}} & -i\sin{\frac{\theta}{2}} \\
        \sin{\frac{\theta}{2}} & \cos{\frac{\theta}{2}} \\
    \end{pmatrix}
\end{equation}

\begin{equation}
    R_z(\theta) = e^{-i \theta Z / 2} = \cos{\frac{\theta}{2}}I - i \sin{\frac{\theta}{2}} Y = \begin{pmatrix}
        e^{-i \theta / 2} & 0 \\
        0 & e^{i\theta/2}
    \end{pmatrix}
\end{equation}

そして、1量子ビットの任意のユンタリ回転ゲートは、これらの$Z-Y$回転行列で分解できる。
実数$\gamma, \phi, \theta, \lambda$を用いて、
\begin{equation}
    U(\theta, \phi, \lambda) = e^{i \gamma} R_z(\phi) R_y(\theta) R_z(\lambda) = e^{i(\gamma - \frac{\theta}{2} - \frac{\phi}{2})}
    \begin{pmatrix}
        \cos{\frac{\theta}{2}} & -e^{i\lambda} \sin{\frac{\theta}{2}} \\
        e^{i\phi}\sin{\frac{\theta}{2}} & e^{i(\lambda + \phi) \cos{\frac{\theta}{2}}}
    \end{pmatrix}
\end{equation}
上記で用いられる$e^{i(\gamma - \frac{\theta}{2} - \frac{\phi}{2})}$は、全体位相と呼ばれ、ブロッホ球上の回転操作や回転角に直接関わることなく、また実際に観測される量ではないため、実際の任意の回転行列は、
\begin{equation}
    \label{eq: hoge}
    U(\theta, \phi, \lambda) = 
    \begin{pmatrix}
        \cos{\frac{\theta}{2}} & -e^{i \lambda}\sin{\frac{\theta}{2}} \\
        e^{i \phi} \sin{\frac{\theta}{2}} & e^{i(\lambda + \phi)} \cos{\frac{\theta}{2}}
    \end{pmatrix}
\end{equation}
となる。

このユニタリ回転ゲートを用いいると、アダマール演算子$H$は、
\begin{align}
    H &= \frac{1}{\sqrt{2}} \begin{pmatrix}
        1 & 1 \\
        1 & -1
    \end{pmatrix} \\
    &= \begin{pmatrix}
        \cos{\frac{\pi}{4}} & -e^{i \pi} \sin{\frac{\pi}{4}} \\
        e^{i 0}\sin{\frac{\pi}{4}} & e^{i \pi} \cos{\frac{\pi}{4}}
    \end{pmatrix} \\
    &= U(\frac{1}{2}\pi, 0, \pi)
\end{align}
となる。

\end{document}
