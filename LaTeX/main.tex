\RequirePackage{plautopatch}
\RequirePackage[l2tabu, orthodox]{nag}

\documentclass[platex,dvipdfmx]{jlreq}			% for platex
% \documentclass[uplatex,dvipdfmx]{jlreq}		% for uplatex
\usepackage{graphicx}
\usepackage{bxtexlogo}
\usepackage{braket}
\usepackage{amsmath}

\title{グローバーのアルゴリズム}

\author{9BSP1118 村岡海人}
\date{\today}
% \begin{document}
% \maketitle
% \section{Cloud LaTeXへようこそ}
\begin{document}
\maketitle

\section{基本的内容}
\subsection{量子ビット}
ビットは古典計算と古典情報の基本概念である。
量子計算と量子情報は類似の概念である量子ビットの上に構築される。


\subsubsection{単一量子ビット}
古典ビットに1あるいは0の状態があるのと同様に、量子ビットも状態を持つ。
量子ビットの2つの可能な状態は$\ket{0} = \begin{pmatrix}
    1 \\
    0\\
\end{pmatrix}$と$\ket{1} = \begin{pmatrix}
    0 \\
    1 \\
\end{pmatrix}$である。
ビットと量子ビットの違いは、量子ビットが$\ket{0}$または、$\ket{1}$以外の状態も取り得ることである。
つまり、状態の線型結合を形作ることがあり、これを重ね合わせと呼ぶ。

\begin{eqnarray}
    \label{eq:superposition}
    \ket{\psi} = \alpha \ket{0} + \beta \ket{1} = \begin{pmatrix}
        \alpha \\
        \beta \\
    \end{pmatrix}
\end{eqnarray}

$\alpha, \beta$は複素数である。
量子ビット状態は2次元複素ベクトル空間のベクトルである。
また、特定の状態$\ket{0}, \ket{1}$を計算基底状態と呼び、このベクトル空間の正規直交基底を構成する。

% 2つの複素数$\alpha, \beta$はどの程度の重みで0と1が重ね合わさってるかを表す複素数である複素確率振幅と呼ばれ、規格化条件$|\alpha|^2 + |\beta|^2 = 1$を満たす。
古典計算では、古典ビットを調べてそれが$0, 1$のいずれの状態にあるかを決めることができる。
しかし、量子ビットを調べてその量子状態、つまり$\alpha, \beta$の値を決めることはできない。
その代わりに、得ることができるのは量子状態に関して施薬された情報だけである。
量子ビットに対して$\ket{0}$と$\ket{1}$のいずれの状態にあるかを調べる測定を行うと、結果は確率$|\aleph|^2$で結果が$0$、または確率$|\beta|^2$で結果が1である。
全確率の和は$1$なので当然$|\alpha|^2 + |\beta|^2 = 1$である。

% ブロッホ球
量子ビットは自由度が2の多くの系で実現されている。
また、量子ビットを外場を用いて制御することができる。
例えば、電子軌道の例の場合、基底状態と第1励起状態が$\ket{0}$と$\ket{1}$に対応し、適切あエネルギーの光を適切な時間照射することで、$\ket{0}$状態から$\ket{1}$状態に変化させることや、その逆を行うことができる。

以下に示すように、ブロッホ球と呼ばれる幾何学的表現が単一量子ビットの状態を考える上で有用である。

$|\alpha|^2 + |\beta|^2 = 1$であるので、式(\ref{eq:superposition})を次のように置き換える。

\begin{eqnarray}
    \ket{\psi} = \cos \frac{\theta}{2} \ket{0} + e^{i \varphi}\sin \frac{\theta}{2} \ket{1}
\end{eqnarray}

ここで、$\theta$と$\varphi$は実数である。
全体にかかる位相因子は観測に影響を与えないで$\ket{0}$の係数を実数とした。
図1に示すように$\theta$と$\varphi$は3次元単位9面上の点を定義する。
この空面をブロッホ球と呼ぶ。
これは単一量子ビット状態を視覚化する便利な方法である。
単一量子ビットの操作はブロッホ球上の描像で記述できる。
しかし、ブロッホ球は多量子ビットに対して一般化できないことに注意する。


\subsection{量子計算}
    
\end{document}
