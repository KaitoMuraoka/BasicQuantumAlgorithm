\RequirePackage{plautopatch}
\RequirePackage[l2tabu, orthodox]{nag}

\documentclass[platex,dvipdfmx]{jlreq}			% for platex
% \documentclass[uplatex,dvipdfmx]{jlreq}		% for uplatex
\usepackage{graphicx}
\usepackage{bxtexlogo}
\usepackage{braket}
\usepackage{amsmath}

\title{グローバーのアルゴリズム}

\author{9BSP1118 村岡海人}
\date{\today}
% \begin{document}
% \maketitle
% \section{Cloud LaTeXへようこそ}
\begin{document}
\maketitle

\section{基本的内容}
\subsection{量子ビット}
ビットは古典計算と古典情報の基本概念である。
量子計算と量子情報はビットの量子版である量子ビットの上に構築される。
\subsubsection{単一量子ビット}
まず量子ビットの説明をする。量子ビットには、古典ビットの0状態、あるいは1状態に対応した、状態$\ket{0}, \ket{1}$ がある。

\begin{eqnarray}
    \begin{pmatrix}
        1 \\
        0 \\
    \end{pmatrix}
    ,
    \begin{pmatrix}
        0 \\
        1 \\
    \end{pmatrix}
\end{eqnarray}

この2つのベクトルは、それぞれ量子ビット$\ket{0}, \ket{1}$に対応する。

\begin{eqnarray}
    \ket{0} =
    \begin{pmatrix}
        1 \\
        0 \\
    \end{pmatrix}
    ,
    \ket{1} = 
    \begin{pmatrix}
        0 \\
        1 \\
    \end{pmatrix}
\end{eqnarray}

1量子ビットは2次元の複素ベクトルで表現される。
従って、2つの基底ベクトルが存在する。


\subsection{量子計算}
    
\end{document}
