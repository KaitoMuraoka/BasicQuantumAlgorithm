\RequirePackage{plautopatch}
\RequirePackage[l2tabu, orthodox]{nag}

\documentclass[platex,dvipdfmx]{jlreq}			% for platex
% \documentclass[uplatex,dvipdfmx]{jlreq}		% for uplatex
\usepackage{graphicx}
\usepackage{bxtexlogo}
\usepackage{braket}
\usepackage{amsmath}

\title{グローバーのアルゴリズム}

\author{9BSP1118 村岡海人}
\date{\today}

\begin{document}
\maketitle

\section{基本的内容}
\subsection{量子ビット}
ビットは古典計算と古典情報の基本概念である。
量子計算と量子情報は類似の概念である量子ビットの上に構築される。


\subsubsection{単一量子ビット}
古典ビットに1あるいは0の状態があるのと同様に、量子ビットも状態を持つ。
量子ビットの2つの可能な状態は$\ket{0} = \begin{pmatrix}
    1 \\
    0\\
\end{pmatrix}$と$\ket{1} = \begin{pmatrix}
    0 \\
    1 \\
\end{pmatrix}$である。
ビットと量子ビットの違いは、量子ビットが$\ket{0}$または、$\ket{1}$以外の状態も取り得ることである。
つまり、状態の線型結合を形作ることがあり、これを重ね合わせと呼ぶ。

\begin{eqnarray}
    \label{eq:superposition}
    \ket{\psi} = \alpha \ket{0} + \beta \ket{1} = \begin{pmatrix}
        \alpha \\
        \beta \\
    \end{pmatrix}
\end{eqnarray}

$\alpha, \beta$は複素数である。
量子ビット状態は2次元複素ベクトル空間のベクトルである。
また、特定の状態$\ket{0}, \ket{1}$を計算基底状態と呼び、このベクトル空間の正規直交基底を構成する。

% 2つの複素数$\alpha, \beta$はどの程度の重みで0と1が重ね合わさってるかを表す複素数である複素確率振幅と呼ばれ、規格化条件$|\alpha|^2 + |\beta|^2 = 1$を満たす。
古典計算では、古典ビットを調べてそれが$0, 1$のいずれの状態にあるかを決めることができる。
しかし、量子ビットを調べてその量子状態、つまり$\alpha, \beta$の値を決めることはできない。
その代わりに、得ることができるのは量子状態に関して施薬された情報だけである。
量子ビットに対して$\ket{0}$と$\ket{1}$のいずれの状態にあるかを調べる測定を行うと、結果は確率$|\aleph|^2$で結果が$0$、または確率$|\beta|^2$で結果が1である。
全確率の和は$1$なので当然$|\alpha|^2 + |\beta|^2 = 1$である。

% ブロッホ球
量子ビットは自由度が2の多くの系で実現されている。
また、量子ビットを外場を用いて制御することができる。
例えば、電子軌道の例の場合、基底状態と第1励起状態が$\ket{0}$と$\ket{1}$に対応し、適切あエネルギーの光を適切な時間照射することで、$\ket{0}$状態から$\ket{1}$状態に変化させることや、その逆を行うことができる。

以下に示すように、ブロッホ球と呼ばれる幾何学的表現が単一量子ビットの状態を考える上で有用である。

$|\alpha|^2 + |\beta|^2 = 1$であるので、式(\ref{eq:superposition})を次のように置き換える。

\begin{eqnarray}
    \ket{\psi} = \cos \frac{\theta}{2} \ket{0} + e^{i \varphi}\sin \frac{\theta}{2} \ket{1}
\end{eqnarray}

ここで、$\theta$と$\varphi$は実数である。
全体にかかる位相因子は観測に影響を与えないで$\ket{0}$の係数を実数とした。
図1に示すように$\theta$と$\varphi$は3次元単位9面上の点を定義する。
この空面をブロッホ球と呼ぶ。
これは単一量子ビット状態を視覚化する便利な方法である。
単一量子ビットの操作はブロッホ球上の描像で記述できる。
しかし、ブロッホ球は多量子ビットに対して一般化できないことに注意する。

\subsubsection{多量子ビット}
次に多量子ビットについて考えてみる。
まず、2つの量子ビットがあるとする。
これが古典ビットならば4つの取り得る状態$00, 01, 10, 11$がある。
これに対応して、2個の量子ビットの系には$\ket{00}, \ket{01}, \ket{10}, \ket{11}$で表される計算基底状態がある。
量子ビットもこれら4つの重ね合せ状態にあるので、2個の量子ビットの量子状態は各計算基底状態に関する複素係数(振幅)を含んでいる。
2個の用紙ビットを記述する状態ベクトルは、

\begin{eqnarray}
    \ket{\psi} = \alpha_{00} \ket{00} + \alpha_{01} \ket{01} + \alpha_{10} \ket{10} + \alpha_{11} \ket{11}
\end{eqnarray}

で与えられる。
% 単一量子ビットの場合と同王に、測定結果$x$($= 00, 01, 10, 11$)は確率$|\alpha_x|^2$で生じ、測定後の量子ビットの状態は$\ket{x}$となる。

% ここで、$\alpha_{00}, \alpha_{01}, \alpha_{10}, \alpha_{11}$はそれぞれの規定の複素確率振幅である。

単一量子ビットの場合と同様に、測定結果$00, 01, 10, 11$はそれぞれの確率$|\alpha_{00}|^2, |\alpha_{01}|^2, |\alpha_{10}|^2, |\alpha_{11}|^2$で生じ、測定後の量子ビットの状態はそれぞれ$\ket{00}, \ket{01}, \ket{10}, \ket{11}$となる。
確率の合計が1になる条件は正規化条件$\sum_{x \in \{0, 1\}^2} |\alpha_x|^2 = 1$で表される。

ここで、記号$\{ 0, 1 \}^2$は「各文字が0または1であり、長さ2の記号列の集合」を意味する。

一般に$n$個の量子ビットを考えると、この系の計算基底状態は$\ket{x_1 x_2 \cdots x_n}$の形をしている。
この系の量子状態は$2^n$個の振幅で規定される。

% ここで$x = x_1 x_2 \cdots x_n$は、$x \in \{ 0, 1 \}^n$であり、$x \in \{ 0, 1 \}^n$は各文字が0か 1であり、長さが$n$の記号例の集合を表す。

\subsection{量子計算}
量子状態に生じている変化は量子計算の言語で記述できる。
古典コンピュータが配線と論理ゲートからなる電気回路で作られているのと同様に、量子コンピュータは量子情報を運び操作するための配線と基本的量子ゲートよりなる量子回路によって作られる。

\subsubsection{単一量子ビットゲート}
古典コンピュータ回路は配線と論理ゲートにより構成されている。
TODO:古典コンピュータの論理ゲートを記述する。


単一量子ビットに作用する量子ゲートは$2 \times 2$の行列で記述できる。

量子ゲートの行列に対する制約について議論する。量子状態$\alpha \ket{0} + \beta \ket{1}$に対して、正規化条件$|\alpha|^2 + |\beta|^2 = 1$が必要であることを考慮すると、正規化条件は
    
\end{document}
